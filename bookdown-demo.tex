\documentclass[]{book}
\usepackage{lmodern}
\usepackage{amssymb,amsmath}
\usepackage{ifxetex,ifluatex}
\usepackage{fixltx2e} % provides \textsubscript
\ifnum 0\ifxetex 1\fi\ifluatex 1\fi=0 % if pdftex
  \usepackage[T1]{fontenc}
  \usepackage[utf8]{inputenc}
\else % if luatex or xelatex
  \ifxetex
    \usepackage{mathspec}
  \else
    \usepackage{fontspec}
  \fi
  \defaultfontfeatures{Ligatures=TeX,Scale=MatchLowercase}
\fi
% use upquote if available, for straight quotes in verbatim environments
\IfFileExists{upquote.sty}{\usepackage{upquote}}{}
% use microtype if available
\IfFileExists{microtype.sty}{%
\usepackage{microtype}
\UseMicrotypeSet[protrusion]{basicmath} % disable protrusion for tt fonts
}{}
\usepackage{hyperref}
\hypersetup{unicode=true,
            pdftitle={Aging Well Lab Manual},
            pdfauthor={Kendra Seaman},
            pdfborder={0 0 0},
            breaklinks=true}
\urlstyle{same}  % don't use monospace font for urls
\usepackage{natbib}
\bibliographystyle{apalike}
\usepackage{longtable,booktabs}
\usepackage{graphicx,grffile}
\makeatletter
\def\maxwidth{\ifdim\Gin@nat@width>\linewidth\linewidth\else\Gin@nat@width\fi}
\def\maxheight{\ifdim\Gin@nat@height>\textheight\textheight\else\Gin@nat@height\fi}
\makeatother
% Scale images if necessary, so that they will not overflow the page
% margins by default, and it is still possible to overwrite the defaults
% using explicit options in \includegraphics[width, height, ...]{}
\setkeys{Gin}{width=\maxwidth,height=\maxheight,keepaspectratio}
\IfFileExists{parskip.sty}{%
\usepackage{parskip}
}{% else
\setlength{\parindent}{0pt}
\setlength{\parskip}{6pt plus 2pt minus 1pt}
}
\setlength{\emergencystretch}{3em}  % prevent overfull lines
\providecommand{\tightlist}{%
  \setlength{\itemsep}{0pt}\setlength{\parskip}{0pt}}
\setcounter{secnumdepth}{5}
% Redefines (sub)paragraphs to behave more like sections
\ifx\paragraph\undefined\else
\let\oldparagraph\paragraph
\renewcommand{\paragraph}[1]{\oldparagraph{#1}\mbox{}}
\fi
\ifx\subparagraph\undefined\else
\let\oldsubparagraph\subparagraph
\renewcommand{\subparagraph}[1]{\oldsubparagraph{#1}\mbox{}}
\fi

%%% Use protect on footnotes to avoid problems with footnotes in titles
\let\rmarkdownfootnote\footnote%
\def\footnote{\protect\rmarkdownfootnote}

%%% Change title format to be more compact
\usepackage{titling}

% Create subtitle command for use in maketitle
\providecommand{\subtitle}[1]{
  \posttitle{
    \begin{center}\large#1\end{center}
    }
}

\setlength{\droptitle}{-2em}

  \title{Aging Well Lab Manual}
    \pretitle{\vspace{\droptitle}\centering\huge}
  \posttitle{\par}
    \author{Kendra Seaman}
    \preauthor{\centering\large\emph}
  \postauthor{\par}
      \predate{\centering\large\emph}
  \postdate{\par}
    \date{2019-08-30}

\usepackage{booktabs}
\usepackage{amsthm}
\makeatletter
\def\thm@space@setup{%
  \thm@preskip=8pt plus 2pt minus 4pt
  \thm@postskip=\thm@preskip
}
\makeatother

\begin{document}
\maketitle

{
\setcounter{tocdepth}{1}
\tableofcontents
}
\hypertarget{introduction}{%
\chapter{Introduction}\label{introduction}}

Welcome to the Aging Well Lab manual! This was created by the Lab Director, Kendra Seaman, to convey my vision for our lab and to communicate community expectations. This manual will be updated regularly as our lab grows and develops. If you have any comments or suggestions regarding the content of this manual, please share these with me. This is a living document, and will change over time as needs arise.

This manual was inspired by other lab manuals, including \href{https://github.com/memobc/memolab-manual}{MemoLab Manual}, \href{http://jpeelle.net/peellelab_manual.pdf}{Peele Lab Manual}, and \href{https://github.com/DVSneuro/smithlab_manual/blob/master/SmithLab_manual.pdf}{Smith Lab Manual}.

\hypertarget{about-the-lab}{%
\section{About the Lab}\label{about-the-lab}}

Our research is dedicated to using basic and translational scientific research studies to promote health and wellbeing across adulthood. We use a variety of behavioral, modeling and neuroimaging techniques to better understand how the mind and the brain change as people get older.

\hypertarget{lab-info}{%
\section{Lab Info}\label{lab-info}}

The lab has several public-facing accounts that anyone can access:

\begin{itemize}
\tightlist
\item
  Website: \url{https://agingwelllab.github.io/}
\item
  GitHub: \url{https://github.com/agingwelllab}
\item
  OSF: \url{https://osf.io/26jqs/}
\end{itemize}

The lab also has sites that are only accessible to lab members:

\begin{itemize}
\tightlist
\item
  CVL Lab Wiki: \url{https://cvlwiki.utdallas.edu/doku.php?id=seamanlab:home}
\item
  Asana: \url{https://app.asana.com}
\item
  Slack: \url{https://agingwelllab.slack.com}
\end{itemize}

\hypertarget{approach}{%
\chapter{Approach}\label{approach}}

We use cognitive modeling and neuroimaging to understand how people learn and make decisions. We are interested how these processes do, \textbf{or do not} change as people get older. The ultimate goal of this research is to promote health and wellbeing across adulthood.

Recognize that this work is inherently interdisciplinary, meaning we will use tools and knowledge from traditional fields like psychology, neuroscience, and economics. Because it requires competency in so many different domains, the learning curve can be steep and it can feel overwhelming. But it is also inherently interesting, and rewarding work!

\hypertarget{mentorship-and-idps}{%
\section{Mentorship and IDPs}\label{mentorship-and-idps}}

To help you with this endeavor, you will be assigned a mentor within the lab. If you are a lab manager, graduate student, or postdoc, the Lab Director will likely be your mentor. If you are an undergraduate, you will likely be reporting to the lab manager, a graduate student, or a postdoc. To facilitate these relationships, we will use individual development plans (IDPs). The structure of IDPs will vary depending on your role in the lab, but generally they will set and track short-term and long-term goals. These will be created when you join the lab and be revisited each semester (i.e.~Fall, Spring, and Summer).

\hypertarget{feedback}{%
\section{Feedback}\label{feedback}}

You should expect to regularly receive feedback from your mentor and your peers. Feedback, especially negative feedback, can be discouraging and overwhelming. Please recognize that the purpose of feedback is to improve your work and help you meet your goals. Also know that giving critical and constructive feedback is an time-consuming effort and \textbf{try} to accept feedback in the spirit in which it is offered. As a group, we will discuss how to give and receive feedback.

\hypertarget{expectations-and-responsibilities}{%
\chapter{Expectations and Responsibilities}\label{expectations-and-responsibilities}}

\hypertarget{everyone}{%
\section{Everyone}\label{everyone}}

\hypertarget{big-picture}{%
\subsection{Big Picture}\label{big-picture}}

We expect everyone to:

\begin{itemize}
\tightlist
\item
  \textbf{Be supportive} - We're all in this together!\\
\item
  \textbf{Share your knowledge.} Mentorship takes many forms, but frequently involves looking out for those who are more junior to us. If you've done something before, share your experience. We are a team and we should work together.
\item
  \textbf{Be engaged in the community.}

  \begin{itemize}
  \tightlist
  \item
    Attend and actively engage in lab and one-on-one meetings. Ask questions, make suggestions, etc. If you are easily distracted by technology, disconnect during lab meetings.\\
  \item
    Attend talks in the CVL, BBS, and greater UTD community.\\
  \item
    Be an positive representative and advocate for our lab and our lab's work in our larger research communities.\\
  \end{itemize}
\item
  Be independent when possible, ask for help when necessary. Specifically, ask three, then me!

  \begin{itemize}
  \tightlist
  \item
    There are lots of great resources on the web you should consult - StackOverflow, NeuroStars, etc\\
  \item
    Use others in the lab (and in the CVL, BBS) and external collaborators.\\
  \end{itemize}
\item
  \textbf{Communicate honestly}, even when it's difficult.\\
\item
  Do work we are proud of individually and as a group.

  \begin{itemize}
  \tightlist
  \item
    Double check your work.\\
  \item
    Our lab has a commitment to open science. Be ready to share your work both within the lab and with outsides at the conclusion of a project.\\
  \end{itemize}
\item
  Work towards proficiency in Unix, BASH, R, and Python.
\item
  Respect each other's strengths, weaknesses, differences, and beliefs.

  \begin{itemize}
  \tightlist
  \item
    Be patient with everyone (including the Lab Director). Most of us are learning new skills and are busier than we would like.
  \end{itemize}
\item
  Adhere to the ethical principles as described by the \href{https://www.apa.org/ethics/code/}{Association for Psychological Science}, \href{https://www.sfn.org/Membership/Professional-Conduct/SfN-Ethics-Policy}{Society for Neuroscience}, and \href{https://research.utdallas.edu/orio/rcr}{UT Dallas Responsible Conduct of Research}.
\item
  Maintain a professional and accurate online presence. Make sure you keep your online profiles up to date. Remember, we all represent the lab and the lab represents us.
\end{itemize}

\hypertarget{small-picture}{%
\subsection{Small Picture}\label{small-picture}}

We're sharing a relatively small space, so please be thoughtful of others. Specifically:

\begin{itemize}
\tightlist
\item
  \textbf{Do not come into the lab if you are sick!} It's better to keep everyone healthy. If you are sick, email your direct supervisor (see organization chart).\\
\item
  Keep the lab neat.

  \begin{itemize}
  \tightlist
  \item
    Do not leave food, drinks, or crumbs in the lab.\\
  \item
    Items left unattended may be cleaned, reclaimed or recycled.
  \end{itemize}
\end{itemize}

\hypertarget{university-policies}{%
\subsection{University policies}\label{university-policies}}

\hypertarget{lab-director}{%
\section{Lab Director}\label{lab-director}}

As the lab director, you can expect me to:

\begin{itemize}
\tightlist
\item
  Have a vision for where the lab is going, both in the short-term (next few weeks) and in the long-term (next few years).
\item
  Obtain funding to support our laboratory.\\
\item
  Care about your happiness.
\item
  Support your career development, including:

  \begin{itemize}
  \tightlist
  \item
    writing recommendation letters,\\
  \item
    introducing you to other scientists (potential future mentors and colleagues),\\
  \item
    promoting your work as often as possible (at conferences),\\
  \item
    facilitating conference travel (see position-dependent specifics below), and\\
  \item
    working with mentees (Postdocs, Mentees) to create an Individual Development Plan (IDP).\\
  \end{itemize}
\item
  Support your personal development, including:

  \begin{itemize}
  \tightlist
  \item
    flexible working hours and environment (when feasible), and\\
  \item
    encouraging activities outside of school/work.\\
  \end{itemize}
\item
  Make the time to meet with you regularly, read and provide feedback on code, posters, manuscripts, and other data products.
\item
  Obsess over chosing the correct analyses, clear phrasing, and awesome data visualizations.
\end{itemize}

\hypertarget{employees}{%
\section{Employees}\label{employees}}

\textbf{Add employee expectations here.}

\hypertarget{lab-manager}{%
\subsection{Lab Manager}\label{lab-manager}}

\begin{itemize}
\tightlist
\item
  Check the voicemail daily and arrange for return calls to be made within one business day.
\end{itemize}

\hypertarget{post-bacc-research-assistants}{%
\subsection{Post-bacc Research Assistants}\label{post-bacc-research-assistants}}

\hypertarget{postdocs-and-staff-scientists}{%
\subsection{Postdocs and Staff Scientists}\label{postdocs-and-staff-scientists}}

I will expect postdocs and staff scientists to move towards being more PI-like, including:

\begin{itemize}
\tightlist
\item
  giving conference talks,\\
\item
  writing grant proposals, and\\
\item
  cultivating an independent research program (up to 10\% of time).
\end{itemize}

Also, to have (or acquire) the technical and open science skills listed for PhD students below.

Postdoc salaries generally follow \href{https://www.niaid.nih.gov/grants-contracts/salary-cap-and-stipend-levels-announced}{NIH guidelines}.

\hypertarget{students}{%
\section{Students}\label{students}}

\textbf{Add expectations for students here.}

\hypertarget{phd-students}{%
\subsection{PhD Students}\label{phd-students}}

I will expect graduate students to:

\begin{itemize}
\tightlist
\item
  attend classes, colloquium, and relevant talks around campus,\\
\item
  be \textbf{excited} about the research questions they are asking, be \textbf{eager} to find the answers, and \textbf{anxious} to share their results with others,\\
\item
  seek out and apply for fellowships and awards (including travel awards), and\\
\item
  realize there are times for pulling all-nighters and times for smelling the roses.
\end{itemize}

I will expect graduate students to move towards:

\begin{itemize}
\tightlist
\item
  expertise in their chosen field(s) by knowing the literature like the back of their hand (see below for suggestions on how to do this),
\item
  proficiency in using R and/or Python for data analysis and model fitting,
\item
  writing BASH shell scripts for imaging analysis in FSL,
\item
  sharing your work with me (and others) using R Markdown and/or Jupyter notebooks,
\item
  preregistering their experiments publicly on OSF,
\item
  sharing their data and scripts publicly on OSF and/or GitHub,
\item
  making figures and posters using R or Python along with Adobe Illustrator,
\item
  clearly communicating your results in written and verbal formats, and
\item
  actively mentoring those working for them (undergraduate RAs), including completing an Individual Development Plan (IDP).
\end{itemize}

The learning curve for these skills can be steep, but developing these skills is necessary for success in both cognitive neuroscience and data science. If these goals do not align with your own interests and goals, then my lab is probably not a good fit for you.

\textbf{Add UTD-specific info here.}

\hypertarget{masters-students}{%
\subsection{Master's Students}\label{masters-students}}

I will expect Master's students to move towards being more PhD student-like. In particular, by the end of their time in the lab, I expect students to:

\begin{itemize}
\tightlist
\item
  complete an empirical study in the lab, and
\item
  complete a poster and/or written manuscript of that study.
\end{itemize}

\hypertarget{undergraduate-students}{%
\subsection{Undergraduate Students}\label{undergraduate-students}}

Undergraduate students will play a vital role in our lab. Students can be involved in the lab in a number of ways, including independent study projects, student works, and internships. Given that we have limited time and resources, unfortunately we cannot accept or keep all undergraduates who are interested in our lab. Based on the lab's needs, we will consider new undergraduates at the beginning of each term. Each new undergraduate must attend a manditory orientation session, journal club, and serve a probationary term (one semester) before advancing further in the lab.

I expect undergraduates to be utterly reliable and willing to help with whatever projects need it. At a bare minimum, reliabillity includes showing up on time, maintaining your hours on the lab calendar, and making sure all of your work is accurate (double-check everything). In most cases, undergraduates will be directly mentored by the lab manager for their first semester in the lab, and then move on to work directly with a research assistant or graduate student.

\hypertarget{communication}{%
\chapter{Communication}\label{communication}}

Communication is critical. Whether you persue academia or industry, you will need to be able to communicate with others about your work.

\hypertarget{communication-within-the-lab}{%
\section{Communication within the lab}\label{communication-within-the-lab}}

\hypertarget{lab-meetings}{%
\subsection{Lab Meetings}\label{lab-meetings}}

Regular lab meetings are essential for making sure that we move both current and future projects forward. They also help us function as a team. We will use lab meetings to discuss administrative and practical issues, solicit feedback on current analyses and code, discuss current literature, and practice conference and job talks. Our weekly lab meetings will run approximately 1-1.5 hours long. All full-time staff and graduate students will be expected to attend (unless cleared with the Lab Director in advance).

During lab meetings, one or two presenters will be responsible for setting the agenda that day. The presenter/s will also be responsible for sharing any relevant materials (e.g.~journal article) a couple of days in advance. If you're scheduled to present and need to cancel or postpone, please give the group 72 hours notice. \textbf{Everyone is expected to participate in lab meetings.} That means reading relevant material beforehand and if you are easily distracted by your computer or phone, put it away.

\hypertarget{individual-meetings}{%
\subsection{Individual Meetings}\label{individual-meetings}}

All managers should have weekly meetings with their mentees, lasting between 15 minutes to an hour. For instance, as lab director, I will have weekly meetings with the lab manager, postdocs, staff scientists, and any students leading projects. Ideally, the agenda for these meetings should be set by the mentee; you should provide a quick overview of the progress you've made on your project and let your mentor know where you're stuck and/or need help. You can also use individual meetings to seek feedback about new project development or professional development

\hypertarget{asana}{%
\subsection{Asana}\label{asana}}

Asana is a project-management platform that we will use to track progress on various projects in the lab.

\begin{itemize}
\tightlist
\item
  Asana is basically an orgnaized to-do list. When adding a task, make sure to put a useful description, assign it to someone and set a tentative due date.\\
\item
  Be realistic and flexible with due dates. Of course you should try to get things done in a timely manner, but sometimes, life happens and you have to adjust.\\
\item
  If you're leading a project in the lab, please make sure that your project is listed on Asana. All project-related tasks should be under your project.
\end{itemize}

\hypertarget{slack}{%
\subsection{Slack}\label{slack}}

Slack is a team-collaboration tool that we will use for communication.

\begin{itemize}
\tightlist
\item
  Try to avoid direct messages in Slack. Instead, find a home for your comment or question in an existing channel. This allowsyou to get answers and feedback from multiple people in the lab.\\
\item
  When replying to a question or comment, try to use threads. This keeps the conversation organized and easy to navigate.
\end{itemize}

\hypertarget{email}{%
\subsection{Email}\label{email}}

When contacting me, please use Slack (or Asana) whenever possible. I will try to respond to emails, but please don't use it for anything urgent.

Likewise, I will try to use Asana/Slack to communicate with you as much as possible. However, sometimes I will need to email you. \textbf{I expect you will read all email sent to you and respond (if a response is needed) within one business day.} If you will not be checking email for more than a couple of days, please consider using a vacation message so that others know you are not available on email (this suggestion also applies to holidays).

The same guideline applies to me: if I don't respond within one business day, please feel free to follow up (but consider using Slack). If I am not available, I will put up a vacation message.

\hypertarget{calendars}{%
\subsection{Calendars}\label{calendars}}

\hypertarget{communication-outside-the-lab}{%
\section{Communication outside the lab}\label{communication-outside-the-lab}}

\hypertarget{phone}{%
\subsection{Phone}\label{phone}}

\begin{itemize}
\item
  If the phone rings in the lab, answer it. Most calls will be from potential (or current) research participants, so it is important to be professional - e.g., ``Aging Well Lab, this is {[}your name{]}. How may I help you?''
\item
  Lab policy is to check the voicemail daily and call back within one business day.
\item
  Please see the lab wiki for specific on speaking to potential research participants.
\end{itemize}

\hypertarget{manuscripts}{%
\subsection{Manuscripts}\label{manuscripts}}

\hypertarget{conference-presentations}{%
\subsection{Conference Presentations}\label{conference-presentations}}

\hypertarget{talks}{%
\subsubsection{Talks}\label{talks}}

\hypertarget{posters}{%
\subsubsection{Posters}\label{posters}}

\hypertarget{open-science-framework}{%
\subsection{Open Science Framework}\label{open-science-framework}}

We will use the Open Science Framework (OSF) for organizing and sharing materials related to our projects. This will include preregistrations, code (sourced from GitHub), posters, and preprints. When making your project page, please follow the guidelines on the main Project page (insert website here).

\hypertarget{github}{%
\subsection{GitHub}\label{github}}

\bibliography{book.bib,packages.bib}


\end{document}
